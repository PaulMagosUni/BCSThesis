\chapter{Conclusione}
\label{conclusione}
%%obiettivo
L'obiettivo della tesi proposta è duplice. Da un lato, è proposta una metodologia per la collezione e l'analisi di dati di Crunchbase. Dall'altro, il lavoro mira a validare i dati di Crunchbase.

%%%summary della pipeline di analisi
Vengono confrontati due metodi di raccolta dati, uno basato sul web scraping e l'altro tramite accesso Accademico all'API. In seguito, l'analisi viene effettuata in primo luogo sull'utenza, per determinare le zone geografiche più rappresentate. Successivamente il focus viene spostato sul confronto dei dati Crunchbase con i dati ufficiali per la validazione. 
Vengono confrontate le scorte di migranti con i dati UN per gli anni 2010, 2015 e 2020 con diversi casi di studio: l'Europa, il nord America, l'Italia e la Gran Bretagna. I flussi Crunchbase ed i flussi di UN ed Eurostat vengono confrontatati per il periodo dal 2010 al 2020, sfruttando diversi livelli di aggregazione al fine di determinare le zone geografiche per cui i flussi Crunchbase sono più significativi. Come per le scorte, vengono dedicati due casi di studio specifici. Il primo riguarda i flussi di migranti in uscita e in entrata dall'Italia, mentre il secondo propone la stessa analisi per la Gran Bretagna. 

%%summary dei risultati - correlazioni non sempre buone , ma a volte sì, e dove
Il confronto dei dati Crunchbase con i dati collezionati dal Custom Query Builder dimostra come lo Scraping pur essendo più immediato può avere delle limitazioni.
L'analisi dell'utenza ha permesso di dedurre che Crunchbase è più rappresentato in nord America,  nord Europa e ovest Europa. 
Le correlazioni ottenute analizzando le scorte Crunchbase con le scorte UN, in generale, presentano una correlazione debole. Tuttavia, si ottengono correlazioni forti quando si osservano gli emigrati di nazionalità italiana e britannica. Si ottengono correlazioni medio-alte anche quando si osserva il confronto delle scorte in Europa e in nord America.
La validazione dei flussi ha determinato che in  generale i flussi Crunchbase non hanno correlazione con i flussi in UN ed Eurostat. Tuttavia, osservando i flussi aggregati in sub continenti si notano correlazioni intorno a 0.5 con Eurostat. L'unione dei due dataset per il confronto dei flussi non apporta miglioramenti particolari alla correlazione con i dati Crunchbase. I casi di studio dedicati all'Italia ed alla Gran Bretagna vedono entrambi valori di correlazione compresi tra 0.5 e 0.6, ma solo per il caso degli emigranti. 

%%implicazioni sulla ricerca -   posso usare questo tipo di dato ma solo per alcuni paesi
I risultati ottenuti indicano che i dati Crunchbase potrebbero essere utilizzati con un certo grado di affidabilità, in studi migratori per gli stati del nord America, e del nord e dell'ovest Europa. In particolare, i casi di studio comuni delle scorte e dei flussi evidenziano una correlazione discreta, se non forte, per gli emigranti italiani e britannici.

%%lavori futuri

Per lavori futuri, si potrebbe analizzare la relazione tra i dati Crunchbase e dati focalizzati sui migranti altamente qualificati, come quelli del Labor Force Survey. Inoltre, dato che Crunchbase detiene informazioni specifiche sugli stati federati degli Stati Uniti (es. Texas, California) si può analizzare la migrazione interna ad essi. Infine, la mobilità degli utenti di Crunchbase potrebbe essere osservata in relazione a eventi particolari (es. Crisi economica, COVID-19) per studiare se e come questa ne venga influenzata.
\newpage
\paragraph{Ringraziamenti}
Un ringraziamento viene fatto in particolare a Crunchbase stessa che ha contribuito accogliendo ed approvando la nostra richiesta di accesso ai dati. Si ringraziano inoltre le relatrici Alina Sîrbu e Laura Pollacci per l'ispirazione e la dottrina impeccabile. 
\centering
\href{https://www.crunchbase.com/}{\includegraphics[width=0.6\textwidth]{images/Crunchbase-Logo.wine.png}}