\chapter*{Introduzione}
\addcontentsline{toc}{chapter}{1 \hspace{5pt} Introduzione}
Lo studio della migrazione riguarda numerosi campi di studio data la sua importanza economica, sociale e per lo sviluppo di nuove politiche. Una categoria di questo fenomeno in particolare è quella della migrazione altamente qualificata ovvero delle persone con un bagaglio di conoscenze di alto livello.
Lo studio di questo fenomeno può portare diversi benefici, come l'analisi dell'integrazione dei migranti altamente qualificati con i nativi della stessa categoria. Ad esempio i dati sugli impieghi possono indicare le aree nelle quali i migranti di questa categoria si specializzino maggiormente \cite{KerrSalriOzden, PeriSharber}. Altri esempi sono lo studio delle competenze dei migranti altamente qualificati in Italia \cite{impicciatore2021emigrazione}, o  gli effetti di avvenimenti di grande portata come la Brexit \cite{doi.org/10.1111/manc.12356}.

Tradizionalmente, gli studi si sono affidati a dati ufficiali raccolti da istituzioni e enti. 
Tuttavia, spesso, i dati ufficiali sono disponibili con ritardo e non sono confrontabili tra stati a causa della mancanza di uno standard nella raccolta \cite{deBeer2010, poulain2006thesim}. 
Recentemente, la disponibilità dei Big Data ha portato la ricerca a sperimentare la possibilità di usare nuove fonti di dati per aggirare e colmare le limitazioni tradizionali. 
Tra le nuove risorse di dati, i cosiddetti social-big data includono diversi tipi di tracce digitali prodotte dalle persone mediante dispositivi mobili, servizi online e piattaforme di social networking \cite{sirbu2021human}. 
Data la complessità del fenomeno migratorio e nonostante gli sforzi della ricerca, molti aspetti possono essere ulteriormente indagati. 
In particolare, grazie a piattaforme di social networking orientate all'ambito lavorativo, come LinkedIn\footnote{LinkedIn \url{https://www.linkedin.com/}} e Crunchbase \footnote{Crunchbase: \url{https://www.crunchbase.com/}}, è possibile osservare informazioni sugli spostamenti dei professionisti attraverso dati non convenzionali. 

Questa tesi propone una metodologia per la collezione e l'estrazione di flussi e scorte migratorio da dati provenienti da Crunchbase, per l'analisi della migrazione altamente qualificata. Inoltre, lo studio mira a valutare l'attendibilità dei dati estratti da Crunchbase mediante il confronto con dati ufficiali, provenienti da Eurostat e United Nations. 




Per la raccolta dati confrontiamo due metodologie diverse. La prima si basa su una procedura di web scraping semi-automatica sul motore di ricerca di Crunchbase, e raccoglie dati aggregati e anonimizzati. La seconda utilizza il Academic Access per ottenere dati su utenti Crunchbase. Per determinare i flussi e le scorte sono state effettuate alcune assunzioni sulla nazionalità possibile di un utente, basandosi sui suoi titoli di studio e sui lavori passati.
Infine, la validazione è stata effettuata attraverso il confronto di scorte e flussi con i rispettivi dati presenti nel Multi-aspect Integrated Migration Indicators (MIMI) dataset \cite{MIMIDOC}. 


Il lavoro svolto ha permesso di acquisire nuove competenze legate all'analisi e la manipolazione del traffico dei siti web, alla programmazione in Python, alla realizzazione di grafici per visualizzare dati complessi come flussi migratori, e al calcolo di correlazioni per i dati visualizzati. Inoltre, sono stati appresi concetti come i flussi migratori e le scorte di migranti, e sono state comprese varie domande di ricerca studiate in ambiti diversi dall'informatica.

Il resto della tesi è strutturato come segue.
Nel Capitolo \ref{StudiMigrazione} vengono introdotti gli argomenti centrali dello studio, con particolare attenzione alla migrazione altamente qualificata. Il capitolo presenta dati e approcci tradizionale (Sezione \ref{DatiTradizionali}) e, successivamente, i dati e metodi non convenzionali (Sezione \ref{DatiNonConvenzionali}). 
Il Capitolo \ref{capitolometodologia} illustra le metodologie e le tecnologie utilizzate per la collezione dei dati, per l'invio di richieste ai server Crunchbase, l'organizzazione dei dati ottenuti attraverso l'accesso accademico e, infine, la validazione dei dati collezionati. 
Il Capitolo \ref{capitoloanalisi} si concentra sulle analisi effettuate, in primo luogo sull'utenza Crunchbase e la sua provenienza. In seguito vengono mostrati e discussi tutti i casi di studio di confronto tra i dati Crunchbase e i dati del MIMI. Alcuni casi di studio si concentrano su alcuni paesi di interesse, come l'Italia e il Regno Unito. 
Infine, il Capitolo \ref{conclusione} conclude l'elaborato discutendo i risultati ottenuti e  le limitazioni incontrate, insieme ai possibili studi futuri.